% -*- mode: latex; mode: auto-fill; coding: utf-8; -*-

\chapter{Conclussion}


%Post Process

%% Storing and rebinding the previous framebuffer means that we can chain
%% several post process nodes together to form complex effects or
%% minimize the cost of applying an effect by using for example separable
%% convolution.

% One/two fbos pr node is a waste but makes it easy to setup and they
% can be configured individually

% Other effects have been implemented, like wobble, edge detection,
% pixelate and an underwater that combines...



\section{Future work}

Future work on the terrain itself would include a clipmap
implementation and level of detail not only on teh textures and
geometry, but also within the shaders. An example would be removing
specular lighting and bumpmapping on distant terrain. An alternative
to the current geomeorphing LOD system is alpha blending, which would
make transistions between LODs easier, but ofcourse requires us to
handle transparency.



An extension to the atmospheric dome is to include the location or
angle of the sun to make a more beautiful sunrise and sunset.
Or going evaen futher an implementing the full atmosphric scattering
effect described in \citebook{chapher~16}{GPUGEMS2}

Create a texture binding manager that reuses \emph{texture unit} by
using \code{glActivateTexture} \citebook{page~440}{redbook} and cache
a stack of the last x number of binded textures.

Texture mapping on the cloud sphere, try with a plane as described in
\citebook{page~32}{copenhagen2007real}


The next step for the \class{PostProcessNode} is to allow it to be
placed anywhere inside the scene graph and then merge the post
processed image with the rest of the scene. Also optimized versions of
the most commen effects could be made to increase performance and in
the case of Depth of Field remove the flickering that comes from only
sampling one focuspoint.

