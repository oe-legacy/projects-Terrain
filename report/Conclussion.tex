% -*- mode: latex; mode: auto-fill; coding: utf-8; -*-

\chapter{Conclussion}


%Post Process

%% Storing and rebinding the previous framebuffer means that we can chain
%% several post process nodes together to form complex effects or
%% minimize the cost of applying an effect by using for example separable
%% convolution.

% One/two fbos pr node is a waste but makes it easy to setup and they
% can be configured individually

% Other effects have been implemented, like wobble, edge detection,
% pixelate and an underwater that combines...



\section{Future work}
Although we have come a long way there are still things that we wish to
do even better. The following techniques have been discussed doing our
work.

The terrain itself could include a clipmap implementation and level of
detail not only on the textures and geometry, but also within the
shaders. An example would be removing specular lighting and
bumpmapping on distant terrain. An alternative to the current
geomeorphing LOD system is alpha blending, which would make
transistions between LODs easier, but ofcourse requires us to handle
transparency.

For our sky we corrently have two minor visual defects: The first
problem is that the atmospheric dome do not take the angle of the sun
into account. This results in a uniform coloring of the dome in the
vertical direction on the entire sphere. A more realistic coloring
would be to only color the dome shades of yellow and red in a radius
around where the sun is setting or rising.
%
We have also talked about going even further an implementing a full
atmospheric scattering simulation algorithm like the
effect described in \citebook{chapher~16}{GPUGEMS2}.
%
The second problem is that the texture mapping of the cloud dome does
not have the right perspective.
To fixed this we have talked about using the texture mapping scheme
based on a plane as described in \citebook{page~32}{copenhagen2007real}.

When generating the cloud texture we use value noise as suggested in
the web page that inspired our
work\footnote{\url{http://freespace.virgin.net/hugo.elias/models/m_perlin.htm}}.
Some authors however use Perlin noise to generate clouds, which could
be interesting to implement and compare with our current value noise.

The next step for the \class{PostProcessNode} is to allow it to be
placed anywhere inside the scene graph and then merge the post
processed image with the rest of the scene. An optimized versions of
some of the most commonly used effects could also be made to increase
performance and in the case of Depth of Field remove the flickering
that comes from only sampling one focus point.

During our work with texture binding the idea of 
creating a texture binding manager that reuses \emph{texture unit}
using \code{glActivateTexture}\citebook{page~440}{redbook} has
surfaced. This manager could cache texture ids and texture units so
the last x number of bound textures more quickly can be activated.
