\chapter{Post Processing}

% Why post processing? Cool effects that can easily be applied to any
% project and reused.

In the previous sections we've had to use geometry to help us create the
landscape, sky and rendering it to the screen. Now we will turn to
screen space effects that requires no geometry, but will instead use
the final color image and the depth buffer to create the
effects. Depth of field, motion blur and a screenwide glow are all
effects that can easily be produced through post processing and there
are many more.

% Easy to use important. Should handle most of the setup

We will try to create a general framework inside OpenEngine that can
handle most effects with the programmer having to do as little setup
as possible. It should also be possible to layer post processing
effects on top of each other, for example to combine a cel shader with
edge detection or optimizing blurring by using a technique called
separable convolution.

\section{Structure}

In this section we will first outline the new scene node called post
process node and how it will help the programmer setup post
processing. Then we will discuss the specifics of rendering the scene,
applying the post process effect to it and how we eventually chose to
structure the rendering.

\subsection*{Post Process Node}

The post process node will apply a fragment program to manipulate the
image rendered while visiting subnodes. Therefore it of course
contains an IShaderResourcePtr that points to the effect. It also
contains a pointer to an OpenEngine FrameBuffer, which the scene is
rendered to, and another FrameBuffer where the scene with the effect
is rendered into. Why the effect is not simply rendered directly to
the backbuffer (or previous frame buffer in case of chained effects)
is explained in the Rendering subsection. The postprocess node also
has a list of ITexture2DPtr's where the final image can be stored
after the effect has been applied, this is useful for some
implementations of motion blur, that relies on the previous image
being availible.

In order to allow users of the PostProcessNode to focus on writing
their effects, the node will handle most of the setup as long as a few
naming conventions are followed. If a uniform named \textit{depth} is
detected then the subscenes depth texture will be bound to this
uniform. Similarly if \textit{imageN}, where \textit{N} is any
integer, is seen, then the FrameBuffer's \textit{N}'th color
attachment will be bound to that uniform. The same goes for the final
image and the uniform name \textit{finalImageN}. The node can also
decide to improve performance by not copying the final image, if this
isn't needed. Of course the previous image doesn't make sense in the
very first rendering, so when initialized the node binds the
not-processed scene image as the final image, and then after the first
frame it switches to the final image. And finally, if neededm the node will pass
the time to the shader.

All of this has allowed us to cut back on doing setup and spend our
time writing cool and fun effects.

\subsection*{Rendering}

% render to a framebuffer

The first thing that needs to be done before switching to the new
framebuffer is to save the old state. 

% TODO can this be done with glPushAttrib?



Once the state has been saved we set the viewport size to the one
specified in the nodes framebuffer and proceed to bind the framebuffer
and clear the color and depth. The subscene is then rendered to the
framebuffer.

% Apply the effect with depth testing set to always

When the scene has been rendered we bind the effect frame buffer and
set OpenGL's depth function to always let fragments pass. We can do
this since we're applying a post process to the entire image and it's
cheaper performance wise than clearing the depth buffer each pass. The
depth test couldn't simply be disabled as that would disable writes to
the depth buffer aswell, and we are interested in keeping the depth
values or modifying them through a post process.

With the effect frame buffer bound, the effect shader and it's
textures can now be applied and rendered onto a square on the entire
screen.

% Store the final image



% Render to the final framebuffer



% Allows multiple postprocess nodes


% Explain the scene frame buffer



\section{Motion Blur}



\section{Glow}



\section{Depth of field}

% GPU gems X

\section{Under Water}

% Combines the previous effects and a wobble

% Create it as several layered effects? Should be possible.


%%% Local Variables:
%%% mode: latex
%%% TeX-master: t
%%% TeX-PDF-mode: t
%%% End:
