% -*- mode: latex; mode: auto-fill; coding: utf-8; -*-

\chapter{Sky}
An essential part of an outdoor environment is the sky, whether it is
night, day, cloudy or clear sky plays an important roll in setting the
mood of the scene. The sky is an ever changing part of any outdoor scene
and depends on many things including both the time of the day and on the
weather conditions.
%
This makes it a hard effect to simulate and visualize, but because it
is such a central part of an outdoor setting, it must be included in
the rendering.

Historically, skies in computer graphics, like many other visual
effects, started of as a simple thing. For a long time, the sky way
simulated by a simple light blue background color, which works fine
when used together with simple scene. But as the other objects in the
scene has become more complex and detailed this simple approach makes the
sky stand out. Therefore developers have started to use sky boxes,
which is a technique that maps images of the sky onto a box. This box
then surrounds the entire scene and the camera is inside the box at
all times, making the illusion of fare stretching planes by using
simple images.
%
One problem with using a box when doing this, is that the edges between the
images on different sides of the box are apparent as artifacts when
looking directly at them. To get around this problem, developers
instead use a sphere or dome as the basic geometry, and hence the name
sky dome, when rendering the sky. The main difference between these
two approaches is the texture mapping of the images. When mapping
images onto a box a straight forward approach can be used, but when
using a dome this step becomes much more tricky.

Returning to the images used when mapping a sky box or dome. When
using stills the sky seams static and dull. By animating ...

\section{Sky domes}
We have chosen to using two sky spheres, one inner sphere for
rendering white clouds with alpha transparency, and an out sphere to
render a background color which can change based on the time of day.
We also include cloud animation based on a simple model for wind, so
the clouds change over time.

\subsection{Geometry}
We have tried two different ways on generating the sphere
geometry. The first method used was the traditional longitudinal/latitude
sphere commonly used for geographical maps. This
however gave very obvious image stretching artifact at the poles
because of oversampling in these regions. Instead we now use geodesic
spheres which consist of equilateral triangles and are beautifully
constructed by recursively subdivision of an icosahedron until the
required LOD is attained.

\subsection{Atmospheric dome}
What phenomena drives the coloring of the sky, making it so beautiful
with it shades of blue and red?
This is the questing ask by researchers studying 
\emph{atmospheric scattering}. This subject has been studied in great detail,
an lots of complex theories and models has been developed within the
field.
We however have chosen not dig deep into this subject because it could
be an entire subject of its own.

However, to color our sky we use a simple model based on color
gradients as described in ref:\footnote{A fast, simple method to
  render sky color using gradients maps, by Jesús Alonso Abad.}
Where the atmospheric dome is colored based on time, height, and the
location of the sun.
We use a color gradient e.g from light to medium blue as a function of
the height and dividing the time into four periods: day, night,
sunrise and sunset, and use the location of the sun the make a
beautiful sunrise and sunset.

\subsection{Cloud Dome}

\subsubsection{Cloud texture}
We have chosen to use a procedural generated image to visualize the
clouds. By take this approach we can change their appearance and
generate very different looking clouds by only altering a few
parameters that is supplied to the generation algorithm.

We have furthermore chosen a three dimensional image because this
eases texture mapping when the image is to be mapped onto the sphere.

\subsubsection{Procedural generated images}
To generate the cloud image we have been inspired by an Internet
homepage\footnote{\url{http://freespace.virgin.net/hugo.elias/models/m_clouds.htm}
  \\Be warned though, this homepage does not use Perlin noise as they
  say but value lattice noise.}
describing how to use procedural generation and
image layering/composition in combination to produce texture that
looks like real clouds.

Procedural generation, is a very fuzzy term, which covers quite a
large area.
--> ref. PG 3rd edition

But boils down to, using some kind of algorithm to produce data, where
a number of parameters can be specified to control the
generation process in a way that alters the result in a intuitive way
based on the parameters.

The basic building block when generating procedural data are noise
functions. A noise function is a function e.g. $f(x)$ which for 
---> ref. PG3rd edition

There are many noise functions, with different properties. But general
they are divide into categories based on how they generation the
noise.

Value lattice noise.

Gradient noise which is the category the includes the most 
now famous noise function known as Perlin noise, first described by Ken
Perlin in famous 1985 paper: Making Noise. 


\subsubsection{Texture mapping - animation}
The basic texture mapping scheme, is to put the three dimensional
texture into a unit cube, and use texture coordinates based on the
unit sphere to make a direct mapping which avoids texture stretching.

If we only did this, the a large portion of the generated texture
would not be used.

So besides the basic texture mapping scheme, we have introduced
texture coordinate animation. This has been done by modeling the wind.







%%% Local Variables:
%%% mode: latex
%%% TeX-master: t
%%% TeX-PDF-mode: t
%%% End:

