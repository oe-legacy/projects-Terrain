\chapter{OpenGL extensions}
\label{sec:GLext}

\section{GL\_EXT\_texture\_array}

The texture array extension does just what its name suggests, it
allows us to store several textures in an array. This is an
improvement in several ways. Firstly we only need to call
glBindTexture only once and then all the textures in the array are at
our disposal. Secondly the GPU only allows a certain number of
textures to be bound, usually 8 or 16. Using a texture array only
takes up one of these slots. In the case of our terrain, using texture
arrays allows us to have an array of 4 color textures; sand, grass,
snow and cliffs. For each of these we're using a normal map to do
bump mapping, which can also be stored in their own texture array. We
can then bind these two arrays, effectively giving us 8 textures at
the price of 2 texture slots.

\section{GL\_ARB\_framebuffer\_object}

FrameBuffers allow programmers to render colors or depth directly to
textures, instead of rendering to the a renderbuffer and then performing
an expensive copy to texture. This is useful when creating post
processing effects such as depth of field or motion blur, since the
effect is a shader applied to the rendered image.

% OE implementation

To facilitate easy setup of frame buffer objects, FBO's, we have
created the \class{FrameBuffer} abstraction in OpenEngine. Users only need to
specify what dimensions their frame buffer object should have, how
many color buffers should be attached, wether or not they want to use
a texture for the depth buffer and then OpenEngine will setup the
entire fbo. This abstraction makes devlopment less error prone to
create FBO's and attach textures to them.


% own dimensions if the programmer wants higher quality images to
% apply the effect to.
