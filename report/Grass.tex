\chapter{Grass}

% GPU gems 1 chapter 7

Now that we have our heightmap rendered, it's time to place some
grass on it. 

Some rendering techniques for grass that where considered were

\begin{itemize}
\item Rendering grass straw textures unto quads. This allows us to
  render several grass straws while only having to process 4
  vertices. Letting it wave in the wind is also easy as that only
  means moving the 2 top vertices. A problem however is that the
  illusion breaks down when viewed form above or on steep cliffs.
\item Modeling each grass straw. This technigue allows for very
  flexible grass with correct lighting. It is however very expensive
  to process all the vertices, so it's ususally used in conjunction
  with the above method for far away grass.
\item Rendering a volume texture containing the grass straws.
\end{itemize}

We opted for grass drawn onto quads, since it's easy to
draw a quad of grass on the screen and grass textures are easy to come
by. 

However we can't simply cover the entire landscape with quads of
grass, as that would be extremely expensive to render, so there are
some design desicions to be made. Another problem with this technique
is that part of the grass texture is transparent, and transparency
isn't easy in the OpenGL 2.1 pipeline. A solution would be depth
peeling or z-sorting the straws, but for our purpose we propose a much
simpler method in the \hyperref[sec:grassFrag]{fragment shader
  subsection}.

% @TODO reference GPU Gems 1

The basis of our grass implementation is Chapter 7 in GPU Gems 1. Here
they propose to draw the quads in objects with star patterns, to
achieve a grass effect independent of the cameras line of sight. Our
implementation does the same, but allows the user to specify how many
quads should be used used pr. star object. Setting this to 1
effectively disables the star pattern and simply renders one quad of
grass straws pr. object instead.

Our animation algorithm is the same as the one proposed in section
7.4.4, where the grass objects center is used to create 'locel
chaos'. This means that we only need one draw call at the moment, we
get minimal texture distortion since the vertices always have the same
distance to their neighbours and we the local chaos creates a more
natural wave effect. Of the two cons to the approach only one of them
really applies to our implementation. The first con, which is that
extra uniform data is required to hold the center of the grass object,
doesn't apply to our case, since we needed the center anyway. We will
explain why in the \hyperref[sec:grassVert]{vertex shader
  subsection}. The second con does apply to our shader, namely that
the complexity of our animation algorithm is limited in order to
minimize the cost of the shader. Our wave therefore is just a simple
cosine function.

As stated earlier we cannot simply draw the grass everywhere without
incurring a huge fps penalty, so instead we focus on drawing the grass
around the camera. %@TODO explain algorithm

\section{Creating the grass}



\section{Rendering}

% One draw call

\subsection{Vertex Shader}\label{sec:grassVert}

% Moving the grass up to the camera

\begin{displaymath}
  (-halfDim, -halfDim) \le vertex + gridDim * n - eye \le (halfDim, halfDim)
\end{displaymath}

\begin{displaymath}
  (0, 0) \le vertex + gridDim * n - eye + halfDim \le (gridDim, gridDim)
\end{displaymath}

\begin{displaymath}
  n = \lfloor (eye - halfDim - vertex) / gridDim + 1 \rfloor
\end{displaymath}

\begin{displaymath}
  newVert = vertex + gridDim * n
\end{displaymath}

% and placing it in front of the camera


% Placing it on the heightmap

% not on sand or snow.

% Making it wave in the wind

% Normal lookup

\subsection{Fragment Shader}\label{sec:grassFrag}

% Discarding seethrough texture (how to avoid z sorting and rendering
% everything with one draw call)

% Simple diffuse lighting, no specular.



% TODO

% 3 lod layers of grass, morphing in, distort the lod layers to avoid
% drawing the same straws (distort by gridsize / PI?)


%%% Local Variables:
%%% mode: latex
%%% TeX-master: t
%%% TeX-PDF-mode: t
%%% End:
