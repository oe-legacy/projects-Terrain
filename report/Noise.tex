% -*- mode: latex; mode: auto-fill; coding: utf-8; -*-

\chapter{Using noise articticly}
Procedural generation, is a very fuzzy term, which covers quite a
large area.
--> ref. PG 3rd edition

But boils down to, using some kind of algorithm to produce data, where
a number of parameters can be specified to control the
generation process in a way that alters the result in a intuitive way
based on the parameters.

The basic building block when generating procedural data are noise
functions. A noise function is a function e.g. $f(x)$ which for 
---> ref. PG3rd edition \citebook{page~123}{ebert2003}

There are many noise functions, with different properties. But general
they are divide into categories based on how they generation the
noise.

Value lattice noise.

Gradient noise which is the category the includes the most 
now famous noise function known as Perlin noise, first described by Ken
Perlin in famous 1985 paper: Making Noise. 



%%% Local Variables:
%%% mode: latex
%%% TeX-master: t
%%% TeX-PDF-mode: t
%%% End:
