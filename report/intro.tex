% -*- mode: latex; mode: auto-fill; coding: utf-8; -*-

\chapter{Introduction}
This report is the product for the second part of the course:
introduction to computer graphics, followed by the authors during the
spring of 2010 at the department of computer science at Aarhus
University. The focus of this second part of the course is OpenGL's
programmable pipeline and OpenGL extensions.
%
To utilize the power of these advanced rendering technologies we have
chosen to try and make a more realistic looking outdoor environment
for one of our previous projects.

\section{Setting the scene}
When creating a scene in computer graphics the director, as when
making a movie or directing a play, writes a script. This script
includes descriptions of the mood and describes the place for where
the story unrolls.
%
This report describes techniques for rendering a scenery 
where the authors had the following setting in mind.
%
The main scene consists of an island surrounded by ocean as far as the
eye can tell, the island has sandy beaches, grass fields, and mountains
with snow on the top.

\section{Previous work}
We have previously been work on this project, but not been satisfied
with the mood of the result because only the basic objects in the
scene was put into play. In this project we want to focus our
attention on the making the details of an outdoor scene seem realistic
and convincing. In our previous work the island was implemented as a
height map, where the ocean was plane separating the terrain into an
underwater environment and land environment. The differentiation of
sandy beach, grass fields, rocks, and snow was based on height and
rendered by interpolating different textures for each type.

\section{Rendering outdoor scenes}
To create a more convincing and realistic outdoor scenery we need to
add more details than done in our previous work. This includes
improving the details of the landscape by using more advanced
lighting, animated grass the waves in the wind, a realistic looking
sky, and added post processing effects.

\section{OpenEngine}
The work in this report as well as our previous work is build on top
of the OpenEngine framework. This framework is an open source project
that was started at Aarhus University in the spring of 2007, and which
was original build to teach the course computer game development (CGD)
the same year. The authors of this report are both deeply involved in
the ongoing development of OpenEngine, and has been so from its
beginning.

OpenEngine is a framework for rendering 3D scene using a scene graph,
the engine is composed of a basic framework, extensions, and
projects. For the work developed in this report, we have one project
called \code{Terrain} that utilizes the basic framework and one extension 
called \code{HeightMap}. The \code{Terrain} project
uses the following other already available extensions: 
\code{SDL} to create a window and get a OpenGL context,
\code{OpenGLRenderer} which includes \code{GLEW} to visit and render
the scene graph,
\code{GenericHandlers} for keyboard and mouse event handling, 
\code{CairoResource} and \code{HUD} to draw and render the FPS counter
in orthogonal projection mode,
\code{AntTweakBar}, \code{Inspection}, and \code{InspectionBar} to add
graphically components which can change values of variables on run-time,

The following extensions is also used in our project but has been
.. expanded during the implementation of our project:
\code{FreeImage} to load and save \code{PNG} and \code{EXR} images, 
\code{MeshUtils} which creates geodesic spheres and other geometric
mesh types, 
\code{TexUtils} which includes our noise generation algorithm
described in chapter \ref{chap:noise},
and \code{OpenGLPostProcessingEffects} which implements the effects
described in chapter \ref{chap:pp}.

% Which extensions contain the important code parts, and include
% filenames in appendix

To use OpenGL's programmable pipeline we have chosen to use the
OpenGL Shader Language (GLSL), we chose this partly because we have
most experience with this shading language and partly because our
previous work already uses this language.

\section{OpenGL version and extensions}
% Antager opengl 2.1 som core så brugte extensions

Since every new version of OpenGL is composed of a set of extension,
OpenGL now consists of quite a large number of extensions. Therefore
instead of listing every extension we're using, which would be quite a
long list, we will instead assume OpenGL 2.1 and only list extensions
not in this version. The extensions used in our project are described
in appendix \ref{sec:GLext}

%%% Local Variables:
%%% mode: latex
%%% TeX-master: t
%%% TeX-PDF-mode: t
%%% End:
