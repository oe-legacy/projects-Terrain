% -*- mode: latex; mode: auto-fill; coding: utf-8; -*-

\chapter{Introduction}
Computer graphics bla. bla. 

\section{Setting the scene}
As when directing a movie or play, the director writes a script, and
sets the scene which sets the mood and describes the place for where
the play / action will unroll.

The following report decribes techniques for ..., where we, the
authors had the following particular setting in mind.

The main scene consists of a island surrounded by ocean as far as the
eye can tell, the island has sandy beaches, grass fields, and mountains
with snow on the top.

This we had allready implemented before the course started. The island
was implemented as a height map, where the ocean was plane seperating
the terrain into and underwater environment and land. The
differantiation of sandy beach, grass fields, rocks, and snow was
based on height and rendered by interpolating different textures for
each type.

\section{Rendering outdoor scenes}
To create a more convincing and realistic outdoor scenery we needed to
add more details than the main ... described by section \ref{}.

The first addition to the above, is to introduce controlable
procedural content.

To convince the audience ... we will introduce motion by animated
clouds, rising and setting sun, waving grass...


Furthermore, we wish to scenery to look astonishing



\section{OpenEngine}

% OpenEngine dependency

% Which extensions contain the important code parts, and include
% filenames in appendix

\section{OpenGL version and extensions}
% Antager opengl 2.1 som core så brugte extensions

Since every new version of OpenGL is composed of a set of extension,
OpenGL now consists of quite a large number of extensions. Therefore
instead of listing every extension we're using, which would be quite a
long list, we will instead assume OpenGL 2.1 and only list extensions
not in this version.

\subsection{GL\_EXT\_texture\_array}

The texture array extension does just what its name suggests, it
allows us to store several textures in an array. This is an
improvement in several ways. Firstly we only need to call
glBindTexture only once and then all the textures in the array are at
our disposal. Secondly the GPU only allows a certain number of
textures to be bound, usually 8 or 16. Using a texture array only
takes up one of these slots. In the case of our terrain, using texture
arrays allows us to have an array of 4 color textures; sand, grass,
snow and cliffs. For each of these we're using a normalmap to do
bumpmapping, which can also be stored in their own texture array. We
can then bind these two arrays, effectively giving us 8 textures at
the price of 2 texture slots.

\subsection{GL\_ARB\_framebuffer\_object}

FrameBuffers allow programmers to render colors or depth directly to
textures, instead of rendering to the a renderbuffer and then performing
an expensive copy to texture. This is useful when creating post
processing effects such as depth of field or motion blur, since the
effect is a shader applied to the rendered image.

% OE implementation

To facilitate easy setup of frame buffer objects, FBO's, we have
created the \class{FrameBuffer} abstraction in OpenEngine. Users only need to
specify what dimensions their frame buffer object should have, how
many color buffers should be attached, wether or not they want to use
a texture for the depth buffer and then OpenEngine will setup the
entire fbo. This abstraction makes devlopment less error prone to
create FBO's and attach textures to them.


% own dimensions if the programmer wants higher quality images to
% apply the effect to.

%%% Local Variables:
%%% mode: latex
%%% TeX-master: t
%%% TeX-PDF-mode: t
%%% End:
