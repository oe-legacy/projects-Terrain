% -*- mode: latex; mode: auto-fill; coding: utf-8; -*-

\chapter{Introduction}
This report is the product of the second part of the course:
Introduction to Computer Graphics, followed by the authors during the
spring of 2010 at the department of computer science at Aarhus
University. The focus of this second part of the course is OpenGL's
programmable pipeline and OpenGL extensions.

To utilize the power of these advanced rendering technologies we have
chosen to try and make a realistic looking outdoor environment. The
project is based on a previous OpenEngine project, but most of it has
been rewritten as part of this project.

\section{Setting the scene}
When creating a scene in computer graphics the director, as when
making a movie or directing a play, writes a script. This script
includes descriptions of the mood and the setting where the story
unfolds.
%
This report describes techniques for rendering a scenery 
where the authors had the following setting in mind.
%
The scene consists of an island surrounded by a calm ocean as far
as the eye can tell. The island has sandy beaches, fields of grass
waving gently in the breeze, and tall mountains with pure white snow
on the top. Clouds are drifting on the breeze and in the horizon
the sun is setting, casting it's last golden rays toward the
landscape.

\section{Previous work}
This project is build ontop of a previous heightmap project in the
OpenEngine framework. While that old project's only purpose was to
explorer heightmap techniques, this project has been about making the
environment more aesthetically pleasing. Alot of effort have therefore
gone into the shaders, both with respect to creating a more realistic
result and optimizing them, this means that none of the old shader code
remains. Most of the code behind the heightmap scene node have also
been completely rewritten to better facilitate extensions made in this
project and for future extensions.

The water used in this project was made by the authors as part of
the old heightmap extension, but contains most of it's original code,
appart from using the new \class{FrameBuffer} abstraction.

Apart from the terrain, we have extended the scene with an
animated skydome, waving grass and post process effects.

%% We have previously been work on this project, but not been satisfied
%% with the mood of the result because only the basic objects in the
%% scene was put into play. In this project we want to focus our
%% attention on the making the details of an outdoor scene seem realistic
%% and convincing. In our previous work the island was implemented as a
%% height map, where the ocean was plane separating the terrain into an
%% underwater environment and land environment. The differentiation of
%% sandy beach, grass fields, rocks, and snow was based on height and
%% rendered by interpolating different textures for each type.

%% \section{Rendering outdoor scenes}
%% To create a more convincing and realistic outdoor scenery we need to
%% add more details than done in our previous work. This includes
%% improving the details of the landscape by using more advanced
%% lighting, animated grass that waves in the wind, a realistic looking
%% sky, and added post processing effects.

\section{OpenEngine}
The work presented in this report is build on top of the OpenEngine
framework. This framework is an open source project that was started
at Aarhus University in the spring of 2007, and was original build to
teach the course Computer Game Development, CGD, the same year. The
authors of this report are both deeply involved in the ongoing
development of OpenEngine, and have been so since its beginning.

OpenEngine is a framework for rendering 3D scenes using a scene graph.
For our project we have mostly implemented new scene graph nodes, such
as \class{HeightmapNode}, \class{MeshNode} or \class{PostProcessNode},
where the two latter were developed during the process of creating
this project and have made their way into the core framework. We use
the two existing nodes; \class{TransformationNode}, for translating
the skydome, and \class{BlendNode}, to make the skydome transparent.\\

OpenEngine is composed of a core framework, extensions, and
projects. In this project we have made use of some already available
extensions:

\begin{itemize}
  \item \code{SDL}, to create a window and get an OpenGL context.
  \item \code{GenericHandlers} for keyboard and mouse event handling.
  \item \code{CairoResource} and \code{HUD} to draw and render the FPS
    counter ontop of the rendered image.
  \item \code{AntTweakBar}, \code{Inspection}, and
    \code{InspectionBar} to add graphical components, allowing us to
    change values of variables at runtime.
\end{itemize}

The following extensions are also used in our project and have either
been created or extended during the implementation of our project:

\begin{itemize}
  \item \code{OpenGLRenderer} which handles all interaction with
    OpenGL inside the OpenEngine framework. This includes the
    allocation of \emph{textures}, \emph{vertex buffer object} and
    \emph{frame buffer objects} on the graphics card and a
    \class{RenderingView} that traverses the scene graph and renders
    it. During this course the OpenEngine core and this extension have
    been extended with buffer object support, three dimensional
    textures, new GLSL shader implementation and our
    \code{PostProcessNode}.
  \item \code{FreeImage} was wrapped in an extension and enables us to
    load and save \code{PNG}, \code{TGA} and \code{EXR} images,
  \item \code{MeshUtils} which was created as part of the new geometry
    structure to handle  buffer objects. It creates the geodesic
    spheres used for the skydomes.
  \item \code{TexUtils} which includes our noise generation algorithm
    described in chapter \ref{chap:noise},
  \item \code{OpenGLPostProcessingEffects} which holds shaders for
    some of the effects described in chapter \ref{chap:pp} and a lot
    of other effects.
  \item \code{Heightmap} which implements a heightmap node and a grass
    node. This extension also includes an expansion to the
    renderingview in \code{OpenGLRenderer}, which can render the
    heightmap and grass.
\end{itemize}

All of these extensions are used by our \code{Terrain} project, which
sets up the scene and holds modules for handling logic, like passing time of
day or sun direction to shaders.

% Which extensions contain the important code parts, and include
% filenames in appendix

To utilise OpenGL's programmable pipeline we chose to use the OpenGL
Shader Language, GLSL, instead of NVIDIA's CG. This was partly due to
our previous experience with GLSL and partly because a lot of work had
gone into optimizing and extending the OpenEngine GLSL shader
implementation in the first quarter of the course.

\section{OpenGL version and extensions}
% Antager opengl 2.1 som core så brugte extensions

Since every new version of OpenGL is composed of a set of extension,
OpenGL now consists of quite a large number of these
extensions. Therefore instead of listing every extension used in this
project, OpenGL 2.1 is assumed and only extensions not in this version
will be listed. The extensions used in our project are described in
appendix \ref{sec:GLext}

%%% Local Variables:
%%% mode: latex
%%% TeX-master: t
%%% TeX-PDF-mode: t
%%% End:
